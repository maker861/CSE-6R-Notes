\documentclass[13pt,a4paper,oneside]{book}
\usepackage[nottoc]{tocbibind}
\usepackage{titlesec}
\usepackage[titles]{tocloft}
\usepackage{amssymb}
\usepackage{ gensymb }
\usepackage{amsthm}
\usepackage{listings}
\usepackage{blkarray}
\usepackage{varwidth}
\usepackage{amsmath}
\usepackage[dvipsnames,svgnames,x11names]{xcolor} % A variety of preset colors
\usepackage[top = 1in, bottom = 1in, left = 1.2in, right=1.2in,marginparwidth=1.2in]{geometry}
\usepackage{enumitem} % There's a built-in \begin{enumerate} ... \end{enumerate}, but this package provides a more customizable one
\usepackage[book]{ragged2e}
\usepackage[most]{tcolorbox}
\newcommand{\N}{\mathbb{N}}
\newcommand{\Z}{\mathbb{Z}}
\newcommand{\Q}{\mathbb{Q}}
\newcommand{\R}{\mathbb{R}}
\newcommand{\I}{\mathbb{I}}
\newcommand{\PP}{\mathbb{P}}
\newtheorem{theorem}{Theorem}
\pagenumbering{arabic}
\usepackage{dsfont}
\usepackage{dcolumn}
\newcolumntype{2}{D{.}{}{2.0}}
\usepackage[dvipsnames]{xcolor}
\usepackage{tikz}
\usepackage{venndiagram}
\usepackage{setspace}
\usepackage[hidelinks]{hyperref}
\hypersetup{colorlinks,
allcolors=black
}

\definecolor{mygreen}{RGB}{28,172,0} % color values Red, Green, Blue
\definecolor{mylilas}{RGB}{170,55,241}
\renewcommand\cftchapfont{\normalfont}
\renewcommand\cftchappagefont{\normalfont}

\AtBeginDocument{\renewcommand\contentsname{Table of Contents}}

\titleformat{\chapter}[display]
{}{\hfill\rule{.7\textwidth}{3pt}}{0pt}
{\hspace*{.3\textwidth}\huge\bfseries}[\addvspace{-1pt}]
\titleformat{name=\chapter,numberless}[display]
{}{\hfill\rule{.7\textwidth}{3pt}}{0pt}
{\hspace*{.3\textwidth}\huge\bfseries}[\addvspace{-1pt}]
\definecolor{codegreen}{rgb}{0,0.6,0}
\definecolor{codegray}{rgb}{0.5,0.5,0.5}
\definecolor{codepurple}{rgb}{0.58,0,0.82}
\definecolor{backcolour}{rgb}{0.95,0.95,0.92}
 
\lstdefinestyle{mystyle}{
    backgroundcolor=\color{backcolour},   
    commentstyle=\color{codegreen},
    keywordstyle=\color{magenta},
    numberstyle=\tiny\color{codegray},
    stringstyle=\color{codepurple},
    basicstyle=\footnotesize,
    breakatwhitespace=false,         
    breaklines=true,                 
    captionpos=b,                    
    keepspaces=true,                 
    numbers=left,                    
    numbersep=5pt,                  
    showspaces=false,                
    showstringspaces=false,
    showtabs=false,                  
    tabsize=2
}
 
\begin{document}
\begin{titlepage}
\begin{center}
	\normalsize
	\begin{doublespace}
	\vspace*{0.5in}
	UC San Diego\
	
	\vspace{1in}
	
	\textbf{\MakeUppercase{CSE6R NOTES}}
	
	\vspace{1in}
	
	Summarized course notes\
	by Emyl Safin bin Mohd Zaky\
	
	
	
	\vspace{0.75in}
	
	
	\vspace{0.75in}
	\end{doublespace} 
	
	\begin{singlespace}
	Instructor:\ 
	Moshiri Niema\
	
	\vfill
	SP23    
	\end{singlespace} 
\end{center}
\end{titlepage}
\frontmatter
\tableofcontents
\clearpage
\mainmatter
\tcbset{
	enhanced,
	colback=backcolour,
	boxrule=0.1pt,
	colframe=red!30!gray,
	fonttitle=\bfseries
}
\lstset{
	style = mystyle
}
\lstset{
language = python
}
\lstset{
	inputpath = D:/Latex/Python/CSE6R/Notes/Code Listing
}
\chapter{The Basics}%
\label{cha:The Basics}
\section{Variables}%
\label{sec:Variables}
We pass data to the computer through the use of \textbf{variables}. Variables can be assigned a value which will have a \textbf{basic data type} and can be plugged into expressions like in math. 
Basic Data Types:
\begin{center}
	int - Holds any \textbf{whole number}.
\end{center}
\lstinputlisting{intexample.py}
\begin{center}
	float - Holds any \textbf{decimal}.
\end{center}
\lstinputlisting{floatexample.py}
\begin{center}
	bool - Holds either \textbf{True} or \textbf{False}.
\end{center}
\lstinputlisting{boolexample.py}
\begin{center}
	str - Holds \textbf{one or more characters}, represented with double or single quotes. (Note that it is common to call variables of type str as strings)
\end{center}
\label{sub:E}
\lstinputlisting{strexample.py}
Note that in Python we can assign multiple variables in one line as shown in the example below.
\lstinputlisting{multivarassign.py}
There is a fifth special basic data type called \textbf{NoneType}.
\begin{center}
	NoneType - When there is no data to be stored in a variable, assigning the value \textbf{None}.
\end{center}
\lstinputlisting{nonetypeexample.py}
\break
\section{Strings}%
\label{sec:Strings}
\subsection{Zero-based numbering}%
\label{subsec:Zero-based numbering}

Consider the string below:
\lstinputlisting{stringindexexample.py}
In Python, data is \textbf{zero-indexed}. Meaning that the \textbf{initial element} of a sequence is assigned the \textbf{position(index) 0} instead of the position 1.\\
Hence in the line above the first element or index 0 is D, the third element or index 2 is u, and so forth.
\subsection{Indexing}%
\label{subsec:Indexing}
We can verify the fact above through the use of the \textbf{indexing operator}, which allows us to access elements at specific indexes. For example:
\lstinputlisting{indexoperatorex.py}
\begin{tcolorbox}[title=Output,center title,hbox]
	\tcbox{D}
\end{tcolorbox}
\subsection*{Substrings}%
\label{sub:Substrings}
We call \textbf{sequential characters} in a string a \textbf{substring}. The indexing operator can be used to extract substrings. In general we do this by:
\lstinputlisting{substringgenex.py}
Here start represents the starting position and end is the ending position of the substring.\\
Note that the $ +1 $ is due to the \textbf{end index} being \textbf{exclusive} in Python.\\
We can see this work more specifically in the example below:
\lstinputlisting{substringspecex.py}
\begin{tcolorbox}[title=Output,center title,hbox]
	\tcbox{Terror}
\end{tcolorbox}
\end{document}
