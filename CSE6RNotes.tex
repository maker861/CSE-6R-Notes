\documentclass[13pt,a4paper,oneside]{book}
\usepackage[nottoc]{tocbibind}
\usepackage{titlesec}
\usepackage[titles]{tocloft}
\usepackage{amssymb}
\usepackage{ gensymb }
\usepackage{amsthm}
\usepackage{listings}
\usepackage{blkarray}
\usepackage{varwidth}
\usepackage{amsmath}
\usepackage[dvipsnames,svgnames,x11names]{xcolor} % A variety of preset colors
\usepackage[top = 1in, bottom = 1in, left = 1.2in, right=1.2in,marginparwidth=1.2in]{geometry}
\usepackage{enumitem} % There's a built-in \begin{enumerate} ... \end{enumerate}, but this package provides a more customizable one
\usepackage[book]{ragged2e}
\usepackage[most]{tcolorbox}
\newcommand{\N}{\mathbb{N}}
\newcommand{\Z}{\mathbb{Z}}
\newcommand{\Q}{\mathbb{Q}}
\newcommand{\R}{\mathbb{R}}
\newcommand{\I}{\mathbb{I}}
\newcommand{\PP}{\mathbb{P}}
\newtheorem{theorem}{Theorem}
\pagenumbering{arabic}
\usepackage{dsfont}
\usepackage{dcolumn}
\newcolumntype{2}{D{.}{}{2.0}}
\usepackage[dvipsnames]{xcolor}
\usepackage{tikz}
\usepackage{venndiagram}
\usepackage{setspace}
\usepackage[hidelinks]{hyperref}
\hypersetup{colorlinks,
allcolors=black
}

\definecolor{mygreen}{RGB}{28,172,0} % color values Red, Green, Blue
\definecolor{mylilas}{RGB}{170,55,241}
\renewcommand\cftchapfont{\normalfont}
\renewcommand\cftchappagefont{\normalfont}

\AtBeginDocument{\renewcommand\contentsname{Table of Contents}}

\titleformat{\chapter}[display]
{}{\hfill\rule{.7\textwidth}{3pt}}{0pt}
{\hspace*{.3\textwidth}\huge\bfseries}[\addvspace{-1pt}]
\titleformat{name=\chapter,numberless}[display]
{}{\hfill\rule{.7\textwidth}{3pt}}{0pt}
{\hspace*{.3\textwidth}\huge\bfseries}[\addvspace{-1pt}]
\definecolor{codegreen}{rgb}{0,0.6,0}
\definecolor{codegray}{rgb}{0.5,0.5,0.5}
\definecolor{codepurple}{rgb}{0.58,0,0.82}
\definecolor{backcolour}{rgb}{0.95,0.95,0.92}
 
\lstdefinestyle{mystyle}{
    backgroundcolor=\color{backcolour},   
    commentstyle=\color{codegreen},
    keywordstyle=\color{magenta},
    numberstyle=\tiny\color{codegray},
    stringstyle=\color{codepurple},
    basicstyle=\footnotesize,
    breakatwhitespace=false,         
    breaklines=true,                 
    captionpos=b,                    
    keepspaces=true,                 
    numbers=left,                    
    numbersep=5pt,                  
    showspaces=false,                
    showstringspaces=false,
    showtabs=false,                  
    tabsize=2
}
 
\begin{document}
\begin{titlepage}
\begin{center}
	\normalsize
	\begin{doublespace}
	\vspace*{0.5in}
	UC San Diego\
	
	\vspace{1in}
	
	\textbf{\MakeUppercase{CSE6R NOTES}}
	
	\vspace{1in}
	
	Summarized course notes\
	by Emyl Safin bin Mohd Zaky\
	
	
	
	\vspace{0.75in}
	
	
	\vspace{0.75in}
	\end{doublespace} 
	
	\begin{singlespace}
	Instructor:\ 
	Moshiri Niema\
	
	\vfill
	SP23    
	\end{singlespace} 
\end{center}
\end{titlepage}
\frontmatter
\tableofcontents
\clearpage
\mainmatter
\tcbset{
	enhanced,
	colback=backcolour,
	boxrule=0.1pt,
	colframe=red!30!gray,
	fonttitle=\bfseries
}
\lstset{
	style = mystyle
}
\lstset{
language = python
}
\lstset{
	inputpath = D:/Latex/Python/CSE6R/Notes/Code Listing
}
\chapter{The Basics}%
\label{cha:The Basics}
\section{Variables}%
\label{sec:Variables}
We pass data to the computer through the use of \textbf{variables}. Variables can be assigned a value which will have a \textbf{basic data type} and can be plugged into expressions like in math. 
Basic Data Types:
\begin{center}
	int - Holds any \textbf{whole number}.
\end{center}
\lstinputlisting{intexample.py}
\begin{center}
	float - Holds any \textbf{decimal}.
\end{center}
\lstinputlisting{floatexample.py}
\begin{center}
	bool - Holds either \textbf{True} or \textbf{False}.
\end{center}
\lstinputlisting{boolexample.py}
\begin{center}
	str - Holds \textbf{one or more characters}, represented with double or single quotes. (Note that it is common to call variables of type str as strings)
\end{center}
\label{sub:E}
\lstinputlisting{strexample.py}
Note that in Python we can assign multiple variables in one line as shown in the example below.
\lstinputlisting{multivarassign.py}
There is a fifth special basic data type called \textbf{NoneType}.
\begin{center}
	NoneType - When there is no data to be stored in a variable, assigning the value \textbf{None}.
\end{center}
\lstinputlisting{nonetypeexample.py}
\break
\section{Strings}%
\label{sec:Strings}
\subsection{Zero-based numbering}%
\label{subsec:Zero-based numbering}

Consider the string below:
\lstinputlisting{stringindexexample.py}
In Python, data is \textbf{zero-indexed}. Meaning that the \textbf{initial element} of a sequence is assigned the \textbf{position(index) 0} instead of the position 1.\\
Hence in the line above the first element or index 0 is D, the third element or index 2 is u, and so forth.
\subsection{Indexing}%
\label{subsec:Indexing}
We can verify the fact above through the use of the \textbf{indexing operator}, which allows us to access elements at specific indexes. For example:
\lstinputlisting{indexoperatorex.py}
\begin{tcolorbox}[title=Output,center title,hbox]
	\tcbox{D}
\end{tcolorbox}
\subsection*{Substrings}%
\label{sub:Substrings}
We call \textbf{sequential characters} in a string a \textbf{substring}. The indexing operator can be used to extract substrings. In general we do this by:
\lstinputlisting{substringgenex.py}
Here start represents the starting position and end is the ending position of the substring.\\
Note that the $ +1 $ is due to the \textbf{end index} being \textbf{exclusive} in Python.\\
We can see this work more specifically in the example below:
\lstinputlisting{substringspecex.py}
\begin{tcolorbox}[title=Output,center title,hbox]
	\tcbox{Terror}
\end{tcolorbox}
\subsection*{String length}%
\label{sub:String length}
We can call the \textbf{len function} to find the \textbf{length} of a string. For example:
\lstinputlisting{lenexample.py}
\begin{tcolorbox}[title=Output,center title,hbox]
	\tcbox{4}
\end{tcolorbox}
It can also be called directly on the string as such:
\lstinputlisting{lengexample.py}
Say we wished to obtain the following about a substring:
\begin{enumerate}
	\item Prefix of string up to index N
	\item Suffix of string starting at index N
\end{enumerate}
We could obtain them as such:
	\lstinputlisting{fixex.py}
Note that I have used shorthand notation above.\\
For the prefix when I exclude the start value Python will assume it is 0, similarly for the suffix when I exclude the end value Python will assume it is the last index $ +1 $.
\subsection*{Negative indices}%
\label{sub:Negative indices}
Negative indices are valid in Python.\\
A \textbf{negative index} $ -n $ represents the $ \mathbf{n} $ \textbf{th character from the end}. For example:
	\lstinputlisting{negindex.py}
	\begin{tcolorbox}[title=Output,center title,hbox]
	\tcbox{2}
	\end{tcolorbox}
\subsection{+ Operator}%
\label{sub:+ Operator}
We can use the \textbf{+ operator} to join multiple strings. For example:
	\lstinputlisting{plusoperatorex.py}
	\begin{tcolorbox}[title=Output,center title,hbox]
	\tcbox{LaTeX}
	\end{tcolorbox}
Strings can only be concatenated with other strings.\\
To concatenate a string with another data type we would get its \textbf{string representation} by calling the function \textbf{str()}. For example:
	\lstinputlisting{strrepex.py}
	\begin{tcolorbox}[title=Output,center title,hbox]
	\tcbox{Pay2Win}
	\end{tcolorbox}
\break
\subsection{Representing long strings}%
\label{sub:Representing long strings}
Say we had a variable that we wanted to assign to a very long string. We could write it out in one line but that would be less legible.\\
Instead we could use \textbf{backslash} "$\backslash$" to \textbf{join multiple strings declared on different lines}. For example:
	\lstinputlisting{backslashex.py}
This would give the same result as declaring the entire string as one.\\
Not important but this behavior is caused by Python's use of implicit concatenation and how Python handles physical and logical lines.
\subsection*{String with multiple lines.}%
\label{sub:String with multiple lines.}
We can store a \textbf{multiline string} into a sigle variable by the use of the $\backslash$\textbf{n} "newlines" each time we wish to have a line break. For example:
	\lstinputlisting{newlineex.py}
Alternatively we could use three quotation marks to improve legibility as shown below:
	\lstinputlisting{triplequoteex.py}
	\begin{tcolorbox}[title=Output,hbox]
		\tcbox[tikznode={align=left}]{
		Number 28.\\\\
		Jackson Pollock, American\\
		1912-1956\\\\
		Enamel on Canvas.
		}
	\end{tcolorbox}
\break
\section{Printing}%
\label{sec:Printing}
To get your program to output text we call the function \textbf{print()}. This is called printing to \textbf{standard output}. \\
We have already seen examples of this function in some of the sample code above. 
To use this function, we just specify whatever you want displayed in the parentheses.
	\lstinputlisting{helloworld.py}
	\begin{tcolorbox}[title=Output,center title,hbox]
	\tcbox{Hello World!}
	\end{tcolorbox}
As seen in earlier examples using print() on a variable will print the value of that variable. It is also important to note that \textbf{indexing and + operators can be used with print()}. For example:
	\lstinputlisting{printex.py}
	\begin{tcolorbox}[title=Output,center title,hbox]
	\tcbox{redrum equals murder}
	\end{tcolorbox}
Note that the $ -1 $ above denotes the step of the indexing operator.
	\lstinputlisting{indexstepex.py}
By default the print() function adds a $\backslash$ n at the end of whatever we are printing.\\
We can change this behavior and determine the character(s) added at the end of print() by:
	\lstinputlisting{printendex.py}
	\begin{tcolorbox}[title=Output,center title,hbox]
		\tcbox{Wraeclast \& Oriath.}
	\end{tcolorbox}
\break
\section{Operators}%
\label{sec:Operators}
\begin{center}
\begin{tabular}{l l l}
	Addition & $+$ &   \\
	Subtraction & - &   \\
	Multiplication & * &   \\
	Division & / &   \\
	Integer division & $/ /$ & $3 / /2=1$ \\
	Remainder (modulo) & \% &   \\
	Exponentiation & ** &   \\
	Negation & - & 
\end{tabular}
\end{center}
\subsection{Operator Precedence}%
\label{sub:Operator Precedence}
From highest to lowest:
\begin{center}
\begin{tabular}{l l}
	$()$ & Parentheses \\
	$* *$ & Exponentiation \\
	$-$ & Negation \\
	$*\,,\,/\,,\,/ /\,,\,\%$ & Multiplication , division , remainder \\
	$+\,,\,-$ & Addition , Subtraction \\
	$=\,,\,+=\,,\,-=\,,\,$ etc. & Assignment \\
\end{tabular}
\end{center}
\section{Comments}%
\label{sec:Comments}
There are two ways we can comment on a piece of code in Python. Single line comments and multi line comments.
\subsection*{Single line comments}%
\label{sub:Single line comments}
We use the character ' \# ' to make single line comments. For example:
	\lstinputlisting{commentsexample.py}
	For single line comments we place the comment \textbf{above} if we're describing a \textbf{block of code} and \textbf{on the same line to the right} if we're describing \textbf{a line of code}.
\subsection*{Multi line comments}%
\label{sub:Multi line c}
We use quote marks to denote multi line comments. A multi line comment starts with three consecutive quote marks and ends with three consecutive quote marks. For example:
	\lstinputlisting{multilinecomment.py}
\chapter{Conditionals}%
\label{cha:Conditionals}
\section{Boolean algebra}%
\label{sec:Boolean algebra}

\end{document}
