\documentclass[13pt,a4paper,oneside]{book}
\usepackage[nottoc]{tocbibind}
\usepackage{titlesec}
\usepackage[titles]{tocloft}
\usepackage{amssymb}
\usepackage{ gensymb }
\usepackage{amsthm}
\usepackage{listings}
\usepackage{blkarray}
\usepackage{amsmath}
\usepackage[dvipsnames,svgnames,x11names]{xcolor} % A variety of preset colors
\usepackage[top = 1in, bottom = 1in, left = 1.2in, right=1.2in,marginparwidth=1.2in]{geometry}
\usepackage{enumitem} % There's a built-in \begin{enumerate} ... \end{enumerate}, but this package provides a more customizable one
\usepackage[book]{ragged2e}
\newcommand{\N}{\mathbb{N}}
\newcommand{\Z}{\mathbb{Z}}
\newcommand{\Q}{\mathbb{Q}}
\newcommand{\R}{\mathbb{R}}
\newcommand{\I}{\mathbb{I}}
\newcommand{\PP}{\mathbb{P}}
\newtheorem{theorem}{Theorem}
\pagenumbering{arabic}
\usepackage{dsfont}
\usepackage{dcolumn}
\newcolumntype{2}{D{.}{}{2.0}}
\usepackage[dvipsnames]{xcolor}
\usepackage{tikz}
\usepackage{venndiagram}
\usepackage{setspace}
\usepackage[hidelinks]{hyperref}
\hypersetup{colorlinks,
allcolors=black
}

\definecolor{mygreen}{RGB}{28,172,0} % color values Red, Green, Blue
\definecolor{mylilas}{RGB}{170,55,241}
\renewcommand\cftchapfont{\normalfont}
\renewcommand\cftchappagefont{\normalfont}

\AtBeginDocument{\renewcommand\contentsname{Table of Contents}}

\titleformat{\chapter}[display]
{}{\hfill\rule{.7\textwidth}{3pt}}{0pt}
{\hspace*{.3\textwidth}\huge\bfseries}[\addvspace{-1pt}]
\titleformat{name=\chapter,numberless}[display]
{}{\hfill\rule{.7\textwidth}{3pt}}{0pt}
{\hspace*{.3\textwidth}\huge\bfseries}[\addvspace{-1pt}]
\definecolor{codegreen}{rgb}{0,0.6,0}
\definecolor{codegray}{rgb}{0.5,0.5,0.5}
\definecolor{codepurple}{rgb}{0.58,0,0.82}
\definecolor{backcolour}{rgb}{0.95,0.95,0.92}
 
\lstdefinestyle{mystyle}{
    backgroundcolor=\color{backcolour},   
    commentstyle=\color{codegreen},
    keywordstyle=\color{magenta},
    numberstyle=\tiny\color{codegray},
    stringstyle=\color{codepurple},
    basicstyle=\footnotesize,
    breakatwhitespace=false,         
    breaklines=true,                 
    captionpos=b,                    
    keepspaces=true,                 
    numbers=left,                    
    numbersep=5pt,                  
    showspaces=false,                
    showstringspaces=false,
    showtabs=false,                  
    tabsize=2
}
 
\begin{document}
\begin{titlepage}
\begin{center}
	\normalsize
	\begin{doublespace}
	\vspace*{0.5in}
	UC San Diego\
	
	\vspace{1in}
	
	\textbf{\MakeUppercase{CSE6R NOTES}}
	
	\vspace{1in}
	
	Summarized course notes\
	by Emyl Safin bin Mohd Zaky\
	
	
	
	\vspace{0.75in}
	
	
	\vspace{0.75in}
	\end{doublespace} 
	
	\begin{singlespace}
	Instructor:\ 
	Moshiri Niema\
	
	\vfill
	SP23    
	\end{singlespace} 
\end{center}
\end{titlepage}
\frontmatter
\tableofcontents
\clearpage
\mainmatter
\lstset{
	style = mystyle
}
\lstset{
language = python
}
\lstset{
	inputpath = D:/Latex/Python/CSE6R/Notes/Code Listing
}
\chapter{The Basics}%
\label{cha:The Basics}
\section{Variables}%
\label{sec:Variables}
We pass data to the computer through the use of \textbf{variables}. Variables can be assigned a value which will have a \textbf{basic data type} and can be plugged into expressions like in math. 
Basic Data Types:
\begin{center}
	int - Holds any \textbf{whole number}.
\end{center}
\subsection*{Example}%
\label{sub:Example}
\lstinputlisting{intexample.py}
\begin{center}
	float - Holds any \textbf{decimal}.
\end{center}
\subsection*{Example}%
\label{sub:Example}
\lstinputlisting{floatexample.py}
\begin{center}
	bool - Holds either \textbf{True} or \textbf{False}.
\end{center}
\lstinputlisting{boolexample.py}
\begin{center}
	str - Holds \textbf{one or more characters}, represented with double or single quotes. (Note that it is common to call variables of type str as strings)
\end{center}
\subsection*{Example}%
\label{sub:E}
\lstinputlisting{strexample.py}
Note that in Python we can assign multiple variables in one line as shown in the example below.
\lstinputlisting{multivarassign.py}
\end{document}
